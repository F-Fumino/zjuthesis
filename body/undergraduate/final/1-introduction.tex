\cleardoublepage

\section{绪论}

\subsection{项目背景和意义}
随着智能制造、建筑仿真、航空航天等工业领域的不断发展,工业场景中的高精度计算机辅助设计(CAD)模型面临着极高的几何复杂度和海量数据处理需求\cite{CAD},这对传统的渲染管线提出了极大的挑战。

在场景导入方面,传统的渲染方法通常要求将当前场景的所有几何数据在渲染前加载至显存。对于数据量庞大的工业场景,这种方法极有可能引发显存溢出,并最终导致系统崩溃。

在实时渲染方面,传统方法中 CPU 需承担剔除计算、场景遍历和渲染排序等任务,且 CPU 与 GPU 之间存在大量协作与通信。面对几何复杂度较高的工业场景,CPU 极易成为性能瓶颈,进而影响 GPU 的高效运行,严重制约渲染性能\cite{WangWei2011}\cite{Tian2024}。与此同时,对于在屏幕空间上占比较小的物体,传统方法仍需渲染其最精细的版本,严重浪费了宝贵的 GPU 资源。

为了应对这些挑战,本项目旨在设计一种基于簇(Cluster/Meshlet)的 GPU 驱动渲染管线,重点研究流式加载(Streaming)、LOD(Level of Detail,细节层次)机制以及剔除计算(Culling)技术,以提高大规模工业场景的导入和实时渲染效率,推动 GPU 渲染在工业领域的进一步发展。

\subsection{同行研发情况}

针对高复杂度场景渲染中的问题,当前最成熟的商业引擎之一——虚幻引擎(Unreal Engine)提出了以 Nanite 为代表的 GPU 驱动渲染管线\cite{Nanite2022}。Nanite 借助虚拟几何(Virtualized Geometry)技术,按需动态加载场景数据,显著降低了实时显存占用。同时,Nanite 采用基于屏幕投影大小的动态 LOD 选择机制\cite{Overton2024},并将剔除计算、LOD 选择等任务从 CPU 转移至 GPU,减轻了 CPU 负担,提高了整体的渲染性能。

尽管 Nanite 在画面质量和渲染性能上具备显著优势,但由于虚幻引擎历史遗留问题较多、系统架构复杂、受限于多平台兼容性,以及用户需求主要集中于游戏领域等因素,其在大规模工业场景中的表现不佳,仍面临预处理时间长、帧率提升不明显、内存和显存占用较高、系统稳定性较差等问题\cite{mattrg2023}。

本项目参考 Nanite 的核心思想,基于自研引擎设计并实现了一套主要面向工业场景的 GPU 驱动渲染管线。与 Nanite 相比,本项目具备以下优势:

\begin{itemize}
    \item 渲染管线更加轻量级,系统架构精简,具备良好的扩展性;
    \item 基于 Vulkan 1.2 进行开发,采用了网格着色器(Mesh Shader)、稀疏资源(Sparse Resources)等 Nanite 未采用的先进技术路径,充分发挥硬件优势,进一步释放渲染性能;
    \item 更加贴合工业场景的需求,导入大规模工业场景的能力比 Nanite 更强,且运行更加稳定。
\end{itemize}

\subsection{本文的组织方式}

本文共分为六章,组织方式如下:

第 1 章为绪论,介绍项目的研究背景和同行研发现状;
第 2 章分析项目需求并提出整体的实施方案;
第 3 至第 5 章重点阐述本人在项目中承担的主要工作,包括基于簇的 GPU 驱动渲染管线的实现,以及 LOD 机制和流式加载模块的设计与实现;
第 6 章分析并总结了项目当前的成果,并对项目未来的改进方向进行了展望。