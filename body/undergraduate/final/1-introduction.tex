\cleardoublepage

\section{绪论}

\subsection{项目背景和意义}
随着智能制造、建筑仿真、航空航天等工业领域的发展,工业场景中的高精度计算机辅助设计(CAD)模型面临着极高的几何复杂度和海量数据处理需求\cite{CAD},这对传统的 CPU 驱动渲染管线提出了极大的挑战。

在场景导入方面,传统方法通常需将整个场景的所有数据一次性加载至显存,对于几何复杂度高、数据量大的工业场景而言,极有可能引发显存溢出并导致应用崩溃。

在实时渲染方面,传统方法 CPU 需承担大量的剔除计算、场景遍历和渲染排序任务。面对工业场景中庞大的模型数量,CPU 极易成为性能瓶颈,进而影响 GPU 的高效运行,严重制约渲染性能\cite{WangWei2011}\cite{Tian2024}。

为了应对这些挑战,本项目旨在设计一种 GPU 驱动的渲染管线,重点研究流式加载(Streaming)、LOD(Level of Detail,细节层次)机制及剔除计算(Culling)技术,以提高大规模工业数据的实时渲染效率,为工业实时可视化、数字孪生和工程仿真等应用提供技术支持,推动 GPU 渲染在工业领域的进一步发展。

\subsection{同行研发情况}

针对大规模场景渲染中的问题,作为当前最成熟的商业引擎之一,虚幻引擎(Unreal Engine)提出了以 Nanite 为代表的 GPU 驱动渲染管线\cite{Nanite2022}。Nanite 通过引入虚拟几何(Virtualized Geometry)技术,按需动态加载资源,显著降低了显存占用。同时,采用基于屏幕占用大小的动态 LOD 机制\cite{Overton2024},并将剔除计算和 LOD 管理等任务从 CPU 转移到 GPU,减轻了 CPU 负担,提高了渲染性能。

尽管 Nanite 在画面质量和渲染性能上具有显著优势,但由于历史遗留问题较多、系统架构复杂、受限于多平台兼容性,以及用户需求主要集中于游戏领域等因素,其在工业场景中的应用仍面临预处理时间长、帧率下降、内存占用高等问题,限制了其在大规模工业场景实时渲染中的表现\cite{mattrg2023}。

本项目参考 Nanite 的核心思想,基于自研引擎设计并实现了一套面向工业场景的 GPU 驱动渲染管线,旨在提升大规模工业场景的实时渲染效率。与 Nanite 相比,本项目设计的渲染管线更加轻量级,精简了系统架构。此外,本项目基于 Vulkan 1.3 进行开发,采用了网格着色器(Mesh Shader)、稀疏资源(Sparse Resources)等 Nanite 未采用的先进技术,充分发挥硬件优势,进一步释放渲染性能。

\subsection{本文的组织方式}

本文共分为六章,具体安排如下:

第 1 章为绪论,介绍项目的背景和同行研发现状;
第 2 章分析项目需求并提出整体实施方案;
第 3 至第 5 章重点阐述本人在项目中承担的主要工作,包括基于簇的 GPU 驱动渲染管线的实现,以及 LOD 机制和流式加载两个关键模块的设计与实现;
第 6 章总结了项目的整体成果,将其与传统渲染管线的运行结果进行对比,并对项目未来的发展方向进行了展望。