\cleardoublepage

\section{绪论}

\subsection{项目背景和意义}
随着智能制造、建筑仿真、航空航天等工业领域的快速发展,工业场景中的高精度计算机辅助设计(CAD)模型面临着极高的几何复杂度和海量数据处理需求,这对传统的 CPU 驱动的渲染管线提出了极大的挑战。

在模型导入方面,传统的渲染方法要求将整个场景一次性加载至显存,这种加载方式很可能引发显存溢出,导致无法成功导入大规模场景。

在实时渲染方面,由于工业场景中对象数量庞大,CPU 需承担大量的剔除计算、场景遍历和渲染排序任务,导致 CPU 负载过高,进而影响 GPU 的高效运行,严重制约渲染性能。

为了应对这些挑战,本项目旨在设计一种基于 GPU 驱动的渲染管线,重点优化流式加载、LOD(Level of Detail,细节层次)机制及剔除计算技术,以提高大规模工业数据的实时渲染效率,为工业实时可视化、数字孪生和工程仿真等应用提供技术支持,推动 GPU 渲染在工业领域的进一步发展。

\subsection{同行研发情况}

针对大规模场景渲染中的问题,作为当前最成熟的商业引擎之一,虚幻引擎(Unreal Engine)提出了以 Nanite 为代表的 GPU 驱动渲染管线。Nanite 通过引入流式加载和虚拟几何技术,按需动态加载资源,显著降低了显存占用。同时,采用基于屏幕占用大小的动态 LOD 机制,
有效优化了渲染负载,并将剔除计算和 LOD 管理等任务从 CPU 转移到 GPU,减轻了 CPU 负担,提高了渲染性能。

尽管 Nanite 在视觉质量和渲染效率上具有显著优势,但由于历史遗留问题较多、系统架构复杂、受限于多平台兼容性,以及用户需求主要集中于游戏领域等因素,其在工业场景中的应用仍面临预处理时间长、帧率下降和内存占用高等问题,限制了其在大规模工业数据可视化和实时渲染中的实际表现。

本项目参考 Nanite 的核心思想,基于自研引擎设计并实现一套面向工业场景的 GPU 驱动渲染管线,旨在提升大规模工业数据的实时渲染效率。与 Nanite 相比,本项目设计的渲染管线更加轻量级,精简了系统架构。此外,本项目基于 Vulkan 1.3 进行开发,采用了网格着色器(Mesh Shader)、稀疏资源(Sparse Resources)等 Nanite 未采用的先进技术,充分发挥硬件优势,进一步释放渲染性能。

\subsection{本文的组织方式}

本文共分为七章,具体安排如下:

第1章为绪论,介绍项目的研究背景和相关研究现状;
第2章分析项目需求并提出整体实施方案;  
第3至第5章重点阐述本人在项目中承担的主要工作,包括基于簇的 GPU 驱动渲染管线的实现,以及 LOD 机制和流式加载两个关键模块的设计与实现;
第6章总结项目成果,并与传统渲染管线的相关结果进行对比,验证该项目在工业大场景下的优越性与实际应用效果。