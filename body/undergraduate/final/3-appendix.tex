\cleardoublepage

{
    \sectionnonum{附~~~~录}

    \appendixsubsecmajornumbering

    \subsection{Vulkan 中的稀疏资源}
    \label{appendix:sparse}

    Vulkan 提供了对稀疏资源的原生支持,允许开发者以页为单位动态地管理资源与显存之间的绑定关系。这一机制类似于虚拟内存系统,能够在不占用完整显存的前提下处理大规模资源,广泛应用于虚拟纹理、虚拟几何等场景中。

    相比于传统资源绑定方式(即非稀疏资源必须一次性绑定整个显存块),稀疏资源大大提高了资源调度的灵活性,并为流式加载等高级技术提供了支持。

    \subsubsection{Vulkan 中的缓冲区类型}

    \begin{figure}[htbp]
        \centering
        \includegraphics[width=\linewidth]{sparse.png}
        \caption{\label{fig:sparse}Vulkan 中不同缓冲区类型的内存绑定方式}
    \end{figure}
    
    如\autoref{fig:sparse}所示,Vulkan 提供了多种缓冲区类型,支持不同形式的稀疏资源管理。根据是否支持稀疏绑定(sparse binding)和稀疏驻留(sparse residency),主要包括以下四类:

    \begin{itemize}
        \item \textbf{Non-sparse Buffer}:默认的缓冲区类型,需在创建后一次性绑定整个显存区域,绑定后不可更改。其简单但缺乏灵活性,无法满足动态资源调度的需求。
        
        \item \textbf{Sparse Binding Buffer}:支持以页为单位绑定显存块,允许在运行时修改缓冲区和显存之间的映射关系。但所有被访问的区域必须已经完成绑定,否则行为未定义。
        
        \item \textbf{Sparse Residency Buffer}:在 Sparse Binding Buffer 的基础上引入了驻留管理机制。允许部分区域未绑定显存,如果访问了未绑定的区域,系统将返回默认值而不会产生非法访问。这类缓冲区非常适合按需加载,支持更大规模的数据组织。
        
        \item \textbf{Sparse Residency Aliased Buffer}:在 Sparse Residency Buffer 的基础上进一步允许多个资源共享同一组显存页,适合需要显存复用的高级应用场景,但需要开发者额外管理资源之间的别名冲突,使用复杂。
    \end{itemize}

    \subsubsection{本项目的选择与理由}
    本项目采用了 Sparse Residency Buffer 来存储大规模工业场景中的几何数据,其主要优势包括:

    \begin{itemize}
        \item \textbf{无需整块分配显存}:支持细粒度的内存控制,避免了为稀疏访问的区域分配完整显存,特别适用于大规模超出 GPU 显存容量的工业 CAD 场景。

        \item \textbf{支持按需加载}:缓冲区可以划分为若干页,仅在需要渲染对应数据时才绑定对应页的显存,减少了不必要的显存占用;

        \item \textbf{支持动态换绑}:允许在运行时重新绑定页面,便于结合页表机制实现灵活的加载与换页策略;
        
        \item \textbf{访问未绑定区域安全}:当部分区域尚未加载时,GPU 访问这些区域也不会触发非法访问,仅返回默认值,这为动态加载和渐进式显示提供了容错能力。
    \end{itemize}

    综上所述,Sparse Residency Buffer 在灵活性与鲁棒性之间达到了较好的平衡,是本项目实现高效流式加载和显存管理的关键基础设施。

    \subsection{源代码清单}

    本项目的全部源代码已托管于 GitHub 开源仓库中,地址如下:

    \begin{itemize}
    \item \url{https://github.com/F-Fumino/Xihe}
    \end{itemize}

    该仓库包含渲染引擎的完整实现,包括基于任务着色器的簇剔除流程、流式加载调度机制、LOD 管理等模块。访问时间为 2025 年 5 月。

    % End of appendix
    \removeappendixsubsecmajornumbering
}